% Options for packages loaded elsewhere
\PassOptionsToPackage{unicode}{hyperref}
\PassOptionsToPackage{hyphens}{url}
%
\documentclass[
  11pt,
]{article}
\usepackage{amsmath,amssymb}
\usepackage{lmodern}
\usepackage{iftex}
\ifPDFTeX
  \usepackage[T1]{fontenc}
  \usepackage[utf8]{inputenc}
  \usepackage{textcomp} % provide euro and other symbols
\else % if luatex or xetex
  \usepackage{unicode-math}
  \defaultfontfeatures{Scale=MatchLowercase}
  \defaultfontfeatures[\rmfamily]{Ligatures=TeX,Scale=1}
\fi
% Use upquote if available, for straight quotes in verbatim environments
\IfFileExists{upquote.sty}{\usepackage{upquote}}{}
\IfFileExists{microtype.sty}{% use microtype if available
  \usepackage[]{microtype}
  \UseMicrotypeSet[protrusion]{basicmath} % disable protrusion for tt fonts
}{}
\makeatletter
\@ifundefined{KOMAClassName}{% if non-KOMA class
  \IfFileExists{parskip.sty}{%
    \usepackage{parskip}
  }{% else
    \setlength{\parindent}{0pt}
    \setlength{\parskip}{6pt plus 2pt minus 1pt}}
}{% if KOMA class
  \KOMAoptions{parskip=half}}
\makeatother
\usepackage{xcolor}
\usepackage[margin=2cm]{geometry}
\usepackage{color}
\usepackage{fancyvrb}
\newcommand{\VerbBar}{|}
\newcommand{\VERB}{\Verb[commandchars=\\\{\}]}
\DefineVerbatimEnvironment{Highlighting}{Verbatim}{commandchars=\\\{\}}
% Add ',fontsize=\small' for more characters per line
\usepackage{framed}
\definecolor{shadecolor}{RGB}{248,248,248}
\newenvironment{Shaded}{\begin{snugshade}}{\end{snugshade}}
\newcommand{\AlertTok}[1]{\textcolor[rgb]{0.94,0.16,0.16}{#1}}
\newcommand{\AnnotationTok}[1]{\textcolor[rgb]{0.56,0.35,0.01}{\textbf{\textit{#1}}}}
\newcommand{\AttributeTok}[1]{\textcolor[rgb]{0.77,0.63,0.00}{#1}}
\newcommand{\BaseNTok}[1]{\textcolor[rgb]{0.00,0.00,0.81}{#1}}
\newcommand{\BuiltInTok}[1]{#1}
\newcommand{\CharTok}[1]{\textcolor[rgb]{0.31,0.60,0.02}{#1}}
\newcommand{\CommentTok}[1]{\textcolor[rgb]{0.56,0.35,0.01}{\textit{#1}}}
\newcommand{\CommentVarTok}[1]{\textcolor[rgb]{0.56,0.35,0.01}{\textbf{\textit{#1}}}}
\newcommand{\ConstantTok}[1]{\textcolor[rgb]{0.00,0.00,0.00}{#1}}
\newcommand{\ControlFlowTok}[1]{\textcolor[rgb]{0.13,0.29,0.53}{\textbf{#1}}}
\newcommand{\DataTypeTok}[1]{\textcolor[rgb]{0.13,0.29,0.53}{#1}}
\newcommand{\DecValTok}[1]{\textcolor[rgb]{0.00,0.00,0.81}{#1}}
\newcommand{\DocumentationTok}[1]{\textcolor[rgb]{0.56,0.35,0.01}{\textbf{\textit{#1}}}}
\newcommand{\ErrorTok}[1]{\textcolor[rgb]{0.64,0.00,0.00}{\textbf{#1}}}
\newcommand{\ExtensionTok}[1]{#1}
\newcommand{\FloatTok}[1]{\textcolor[rgb]{0.00,0.00,0.81}{#1}}
\newcommand{\FunctionTok}[1]{\textcolor[rgb]{0.00,0.00,0.00}{#1}}
\newcommand{\ImportTok}[1]{#1}
\newcommand{\InformationTok}[1]{\textcolor[rgb]{0.56,0.35,0.01}{\textbf{\textit{#1}}}}
\newcommand{\KeywordTok}[1]{\textcolor[rgb]{0.13,0.29,0.53}{\textbf{#1}}}
\newcommand{\NormalTok}[1]{#1}
\newcommand{\OperatorTok}[1]{\textcolor[rgb]{0.81,0.36,0.00}{\textbf{#1}}}
\newcommand{\OtherTok}[1]{\textcolor[rgb]{0.56,0.35,0.01}{#1}}
\newcommand{\PreprocessorTok}[1]{\textcolor[rgb]{0.56,0.35,0.01}{\textit{#1}}}
\newcommand{\RegionMarkerTok}[1]{#1}
\newcommand{\SpecialCharTok}[1]{\textcolor[rgb]{0.00,0.00,0.00}{#1}}
\newcommand{\SpecialStringTok}[1]{\textcolor[rgb]{0.31,0.60,0.02}{#1}}
\newcommand{\StringTok}[1]{\textcolor[rgb]{0.31,0.60,0.02}{#1}}
\newcommand{\VariableTok}[1]{\textcolor[rgb]{0.00,0.00,0.00}{#1}}
\newcommand{\VerbatimStringTok}[1]{\textcolor[rgb]{0.31,0.60,0.02}{#1}}
\newcommand{\WarningTok}[1]{\textcolor[rgb]{0.56,0.35,0.01}{\textbf{\textit{#1}}}}
\usepackage{graphicx}
\makeatletter
\def\maxwidth{\ifdim\Gin@nat@width>\linewidth\linewidth\else\Gin@nat@width\fi}
\def\maxheight{\ifdim\Gin@nat@height>\textheight\textheight\else\Gin@nat@height\fi}
\makeatother
% Scale images if necessary, so that they will not overflow the page
% margins by default, and it is still possible to overwrite the defaults
% using explicit options in \includegraphics[width, height, ...]{}
\setkeys{Gin}{width=\maxwidth,height=\maxheight,keepaspectratio}
% Set default figure placement to htbp
\makeatletter
\def\fps@figure{htbp}
\makeatother
\setlength{\emergencystretch}{3em} % prevent overfull lines
\providecommand{\tightlist}{%
  \setlength{\itemsep}{0pt}\setlength{\parskip}{0pt}}
\setcounter{secnumdepth}{-\maxdimen} % remove section numbering
\ifLuaTeX
  \usepackage{selnolig}  % disable illegal ligatures
\fi
\IfFileExists{bookmark.sty}{\usepackage{bookmark}}{\usepackage{hyperref}}
\IfFileExists{xurl.sty}{\usepackage{xurl}}{} % add URL line breaks if available
\urlstyle{same} % disable monospaced font for URLs
\hypersetup{
  pdftitle={SML HW3},
  pdfauthor={Yujia Guo \& Christopher Marais},
  hidelinks,
  pdfcreator={LaTeX via pandoc}}

\title{SML HW3}
\author{Yujia Guo \& Christopher Marais}
\date{}

\begin{document}
\maketitle

\hypertarget{problem-1}{%
\section{Problem 1}\label{problem-1}}

\begin{Shaded}
\begin{Highlighting}[]
\NormalTok{x1 }\OtherTok{\textless{}{-}} \FunctionTok{c}\NormalTok{(}\DecValTok{6}\NormalTok{,}\DecValTok{2}\NormalTok{,}\DecValTok{9}\NormalTok{,}\DecValTok{1}\NormalTok{)}
\NormalTok{x2 }\OtherTok{\textless{}{-}} \FunctionTok{c}\NormalTok{(}\DecValTok{2}\NormalTok{,}\DecValTok{4}\NormalTok{,}\DecValTok{1}\NormalTok{,}\DecValTok{2}\NormalTok{)}
\NormalTok{x3 }\OtherTok{\textless{}{-}} \FunctionTok{c}\NormalTok{(}\DecValTok{4}\NormalTok{,}\SpecialCharTok{{-}}\DecValTok{2}\NormalTok{,}\DecValTok{8}\NormalTok{,}\SpecialCharTok{{-}}\DecValTok{1}\NormalTok{)}
\NormalTok{m1 }\OtherTok{\textless{}{-}} \FunctionTok{cbind}\NormalTok{(x1,x2,x3)}
\NormalTok{lmod }\OtherTok{\textless{}{-}} \FunctionTok{lm}\NormalTok{(x1}\SpecialCharTok{\textasciitilde{}}\NormalTok{x3}\SpecialCharTok{+}\NormalTok{x2)}
\FunctionTok{summary}\NormalTok{(lmod)}
\end{Highlighting}
\end{Shaded}

\begin{verbatim}
## Warning in summary.lm(lmod): essentially perfect fit: summary may be unreliable
\end{verbatim}

\begin{verbatim}
## 
## Call:
## lm(formula = x1 ~ x3 + x2)
## 
## Residuals:
##          1          2          3          4 
## -1.846e-16  5.430e-17  1.086e-16  2.172e-17 
## 
## Coefficients:
##              Estimate Std. Error   t value Pr(>|t|)    
## (Intercept) 4.441e-16  4.964e-16 8.950e-01    0.535    
## x3          1.000e+00  4.734e-17 2.112e+16   <2e-16 ***
## x2          1.000e+00  1.748e-16 5.721e+15   <2e-16 ***
## ---
## Signif. codes:  0 '***' 0.001 '**' 0.01 '*' 0.05 '.' 0.1 ' ' 1
## 
## Residual standard error: 2.22e-16 on 1 degrees of freedom
## Multiple R-squared:      1,  Adjusted R-squared:      1 
## F-statistic: 4.158e+32 on 2 and 1 DF,  p-value: < 2.2e-16
\end{verbatim}

x1, x2, x3 are linearly dependent. Regress x1 on x2 and x3, we will get
that the R-squared equals to 1 and RSS=0. Also, since x1 is depend on
both x1 and x2, the regression p-value of x2 and x3 are both less than
0.05. In general, A set of vectors \{x1, x2, · · · , xp\} is linearly
dependent if there exist numbers β1, β2, · · · βp ,not all equal to
zero, such that β1x1 + β2x2 + · · · + βpxP = 0 It can be re-written as
x1 = −(β2x2 + · · · + βpxP )/β1 Therefore, we can get a perfect fit and
the rss will be 0.

\hypertarget{problem-2}{%
\section{Problem 2}\label{problem-2}}

\begin{Shaded}
\begin{Highlighting}[]
\NormalTok{m2 }\OtherTok{\textless{}{-}} \FunctionTok{matrix}\NormalTok{(}\FunctionTok{rnorm}\NormalTok{(}\DecValTok{10000}\SpecialCharTok{*} \DecValTok{2}\NormalTok{), }\CommentTok{\# Create random matrix}
\AttributeTok{ncol =} \DecValTok{2}\NormalTok{)}
\FunctionTok{cor}\NormalTok{(m2[,}\DecValTok{1}\NormalTok{],m2[,}\DecValTok{2}\NormalTok{])}
\end{Highlighting}
\end{Shaded}

\begin{verbatim}
## [1] 0.01748097
\end{verbatim}

The correlation of this sample is very closed to 0. We can conclude that
in general, if numerical predictors are linearly independent, then they
are uncorrelated.

\hypertarget{problem-3}{%
\section{Problem 3}\label{problem-3}}

\hypertarget{section}{%
\subsection{3.1}\label{section}}

\begin{Shaded}
\begin{Highlighting}[]
\NormalTok{myOLS }\OtherTok{\textless{}{-}} \ControlFlowTok{function}\NormalTok{(Y, X, }\AttributeTok{is1 =} \ConstantTok{TRUE}\NormalTok{) \{}
\CommentTok{\# Inputs:}
\CommentTok{\# * Y is the vector of length n of response variables}
\CommentTok{\# * X is an n{-}by{-}p matrix of numerical covariates (in columns); p \textless{} n}
\CommentTok{\# ** assume the columns of X are linearly independent and}
\CommentTok{\# ** do not include the column for the intercept as a part of the X matrix}
\CommentTok{\# * is1 is a logical "flag" whether the intercept is included; is1 = TRUE by default}
\CommentTok{\# Output:}
\CommentTok{\# the function must return a list L with two elements:}
\CommentTok{\# L[1] will contain the vector of OLS/MLE coefficients, betahat}
\CommentTok{\# L[2] will contain standard errors (i.e., estimated standard deviations) for betahat}

\NormalTok{  n\_y }\OtherTok{=} \FunctionTok{length}\NormalTok{(Y)}
\NormalTok{  n\_x }\OtherTok{=} \FunctionTok{nrow}\NormalTok{(X)}
  \ControlFlowTok{if}\NormalTok{(n\_x }\SpecialCharTok{!=}\NormalTok{ n\_y)\{}
    \FunctionTok{print}\NormalTok{(}\StringTok{"Error: X and Y not of same length"}\NormalTok{)}
\NormalTok{  \}}\ControlFlowTok{else}\NormalTok{\{}
\NormalTok{    n }\OtherTok{=}\NormalTok{ n\_y}
\NormalTok{  \}}
  
\NormalTok{  p }\OtherTok{=} \FunctionTok{ncol}\NormalTok{(X)}
  
  \ControlFlowTok{if}\NormalTok{(is1}\SpecialCharTok{==}\ConstantTok{FALSE}\NormalTok{)\{}
    \CommentTok{\# add intercept to matrix}
\NormalTok{    X0 }\OtherTok{=} \FunctionTok{rep}\NormalTok{(}\DecValTok{1}\NormalTok{, n)}
\NormalTok{    X }\OtherTok{=} \FunctionTok{cbind}\NormalTok{(X0, X)}
\NormalTok{  \}}
  
\NormalTok{  betahat }\OtherTok{=} \FunctionTok{solve}\NormalTok{(}\FunctionTok{t}\NormalTok{(X)}\SpecialCharTok{\%*\%}\NormalTok{X)}\SpecialCharTok{\%*\%}\FunctionTok{t}\NormalTok{(X)}\SpecialCharTok{\%*\%}\NormalTok{Y}
\NormalTok{  pred }\OtherTok{=}\NormalTok{ X}\SpecialCharTok{\%*\%}\NormalTok{betahat }\CommentTok{\# prediction}
\NormalTok{  sigma\_sq }\OtherTok{\textless{}{-}} \FunctionTok{sum}\NormalTok{((Y }\SpecialCharTok{{-}}\NormalTok{ pred)}\SpecialCharTok{\^{}}\DecValTok{2}\NormalTok{)}\SpecialCharTok{/}\NormalTok{(n}\SpecialCharTok{{-}}\NormalTok{p)  }\CommentTok{\# estimate of sigma{-}squared}
\NormalTok{  var\_covar }\OtherTok{\textless{}{-}}\NormalTok{ sigma\_sq}\SpecialCharTok{*}\FunctionTok{solve}\NormalTok{(}\FunctionTok{t}\NormalTok{(X)}\SpecialCharTok{\%*\%}\NormalTok{X) }\CommentTok{\# variance covariance matrix}
\NormalTok{  std\_err }\OtherTok{\textless{}{-}} \FunctionTok{sqrt}\NormalTok{(}\FunctionTok{diag}\NormalTok{(var\_covar)) }\CommentTok{\# standard error}
  
  \FunctionTok{return}\NormalTok{(}\FunctionTok{list}\NormalTok{(betahat, std\_err))}
\NormalTok{\}}
\end{Highlighting}
\end{Shaded}

\begin{Shaded}
\begin{Highlighting}[]
\NormalTok{n }\OtherTok{=} \DecValTok{30}
\FunctionTok{set.seed}\NormalTok{(}\DecValTok{0}\NormalTok{)}
\NormalTok{p }\OtherTok{=} \DecValTok{3}
\NormalTok{X }\OtherTok{=} \FunctionTok{matrix}\NormalTok{(}\FunctionTok{runif}\NormalTok{(n}\SpecialCharTok{*}\NormalTok{p),}\AttributeTok{nrow=}\NormalTok{n)}\SpecialCharTok{*}\DecValTok{2{-}1}
\NormalTok{b }\OtherTok{=} \FunctionTok{seq}\NormalTok{(}\DecValTok{1}\NormalTok{,p,}\AttributeTok{by=}\DecValTok{1}\NormalTok{)}
\NormalTok{Y }\OtherTok{=}\NormalTok{ X}\SpecialCharTok{\%*\%}\NormalTok{b }\SpecialCharTok{+} \FunctionTok{rnorm}\NormalTok{(n)}
\NormalTok{fit1 }\OtherTok{=} \FunctionTok{lm}\NormalTok{(Y }\SpecialCharTok{\textasciitilde{}}\NormalTok{ X); }\FunctionTok{summary}\NormalTok{(fit1)}
\end{Highlighting}
\end{Shaded}

\begin{verbatim}
## 
## Call:
## lm(formula = Y ~ X)
## 
## Residuals:
##     Min      1Q  Median      3Q     Max 
## -2.3701 -0.3304  0.1082  0.4938  2.3930 
## 
## Coefficients:
##             Estimate Std. Error t value Pr(>|t|)    
## (Intercept)  -0.1291     0.1708  -0.756  0.45648    
## X1            0.9214     0.2987   3.085  0.00478 ** 
## X2            2.4021     0.3468   6.926 2.36e-07 ***
## X3            2.7482     0.3452   7.960 1.94e-08 ***
## ---
## Signif. codes:  0 '***' 0.001 '**' 0.01 '*' 0.05 '.' 0.1 ' ' 1
## 
## Residual standard error: 0.9235 on 26 degrees of freedom
## Multiple R-squared:  0.802,  Adjusted R-squared:  0.7792 
## F-statistic: 35.11 on 3 and 26 DF,  p-value: 2.711e-09
\end{verbatim}

\begin{Shaded}
\begin{Highlighting}[]
\FunctionTok{myOLS}\NormalTok{(Y,X,}\ConstantTok{FALSE}\NormalTok{)}
\end{Highlighting}
\end{Shaded}

\begin{verbatim}
## [[1]]
##          [,1]
## X0 -0.1291359
##     0.9214173
##     2.4020865
##     2.7482181
## 
## [[2]]
##        X0                               
## 0.1676341 0.2930771 0.3403307 0.3387858
\end{verbatim}

\hypertarget{section-1}{%
\subsection{3.2}\label{section-1}}

\begin{Shaded}
\begin{Highlighting}[]
\NormalTok{myPolyReg1 }\OtherTok{\textless{}{-}} \ControlFlowTok{function}\NormalTok{(Y, X1, }\AttributeTok{deg=}\DecValTok{1}\NormalTok{) \{}
\CommentTok{\# Inputs: same as for myOLS, except}
\CommentTok{\# * X1 is a vector of length n that contain the covariate values (numerical)}
\CommentTok{\# * deg is the degree k (i.e., largest power) of the polynomial fit; k \textless{} n ; deg=1 by default.}
\CommentTok{\# Outputs: same as for myOLS}
\NormalTok{  X }\OtherTok{=} \FunctionTok{rep}\NormalTok{(}\DecValTok{1}\NormalTok{, n)}
  \ControlFlowTok{for}\NormalTok{ (i }\ControlFlowTok{in} \FunctionTok{seq}\NormalTok{(}\DecValTok{1}\NormalTok{, deg))\{}
\NormalTok{    X }\OtherTok{=} \FunctionTok{cbind}\NormalTok{(X, X1}\SpecialCharTok{**}\NormalTok{i)}
\NormalTok{  \}}
  
  \FunctionTok{return}\NormalTok{(}\FunctionTok{myOLS}\NormalTok{(}\AttributeTok{X=}\NormalTok{X, }\AttributeTok{Y=}\NormalTok{Y, }\AttributeTok{is1=}\ConstantTok{TRUE}\NormalTok{))}
\NormalTok{\}}
\end{Highlighting}
\end{Shaded}

\begin{Shaded}
\begin{Highlighting}[]
\NormalTok{n }\OtherTok{=} \DecValTok{30} 
\FunctionTok{set.seed}\NormalTok{(}\DecValTok{0}\NormalTok{)}
\NormalTok{X }\OtherTok{=} \FunctionTok{runif}\NormalTok{(n)}\SpecialCharTok{*}\DecValTok{4{-}2} \CommentTok{\# X is uniformly distributed on [{-}2,2]}
\NormalTok{Y }\OtherTok{=} \DecValTok{1} \SpecialCharTok{+} \DecValTok{3}\SpecialCharTok{*}\NormalTok{X }\SpecialCharTok{{-}}\DecValTok{2}\SpecialCharTok{*}\NormalTok{X}\SpecialCharTok{\^{}}\DecValTok{2} \SpecialCharTok{+} \DecValTok{1}\SpecialCharTok{*}\NormalTok{X}\SpecialCharTok{\^{}}\DecValTok{3} \SpecialCharTok{+} \FunctionTok{rnorm}\NormalTok{(n)}
\NormalTok{fit0 }\OtherTok{=} \FunctionTok{lm}\NormalTok{(Y }\SpecialCharTok{\textasciitilde{}}\NormalTok{ X }\SpecialCharTok{+} \FunctionTok{I}\NormalTok{(X}\SpecialCharTok{\^{}}\DecValTok{2}\NormalTok{) }\SpecialCharTok{+} \FunctionTok{I}\NormalTok{(X}\SpecialCharTok{\^{}}\DecValTok{3}\NormalTok{))}
\FunctionTok{summary}\NormalTok{(fit0)}
\end{Highlighting}
\end{Shaded}

\begin{verbatim}
## 
## Call:
## lm(formula = Y ~ X + I(X^2) + I(X^3))
## 
## Residuals:
##     Min      1Q  Median      3Q     Max 
## -1.3159 -0.5052 -0.1633  0.5612  1.6267 
## 
## Coefficients:
##             Estimate Std. Error t value Pr(>|t|)    
## (Intercept)   1.1415     0.2385   4.787 5.90e-05 ***
## X             2.9127     0.3166   9.199 1.17e-09 ***
## I(X^2)       -2.0916     0.1318 -15.866 6.88e-15 ***
## I(X^3)        1.0150     0.1221   8.313 8.56e-09 ***
## ---
## Signif. codes:  0 '***' 0.001 '**' 0.01 '*' 0.05 '.' 0.1 ' ' 1
## 
## Residual standard error: 0.8295 on 26 degrees of freedom
## Multiple R-squared:  0.9856, Adjusted R-squared:  0.984 
## F-statistic: 594.4 on 3 and 26 DF,  p-value: < 2.2e-16
\end{verbatim}

\begin{Shaded}
\begin{Highlighting}[]
\FunctionTok{myPolyReg1}\NormalTok{(Y,X,}\AttributeTok{deg=}\DecValTok{3}\NormalTok{)}
\end{Highlighting}
\end{Shaded}

\begin{verbatim}
## [[1]]
##        [,1]
## X  1.141535
##    2.912750
##   -2.091594
##    1.015042
## 
## [[2]]
##         X                               
## 0.2384849 0.3166353 0.1318304 0.1221091
\end{verbatim}

\hypertarget{section-2}{%
\subsection{3.3}\label{section-2}}

\begin{Shaded}
\begin{Highlighting}[]
\NormalTok{myAnova1 }\OtherTok{\textless{}{-}} \ControlFlowTok{function}\NormalTok{(Y, XF, }\AttributeTok{is1=}\ConstantTok{TRUE}\NormalTok{) \{}
\CommentTok{\# Inputs: same as for myOLS, except}
\CommentTok{\# * XF is a vector of length n that contain the covariate values (categorical or "factor")}
\CommentTok{\# Outputs: same as for myOLS}
\NormalTok{  uniq\_var }\OtherTok{\textless{}{-}} \FunctionTok{unique}\NormalTok{(XF)}
\NormalTok{  X }\OtherTok{\textless{}{-}} \SpecialCharTok{+}\FunctionTok{outer}\NormalTok{(XF, uniq\_var, }\StringTok{\textasciigrave{}}\AttributeTok{==}\StringTok{\textasciigrave{}}\NormalTok{)}
  \FunctionTok{colnames}\NormalTok{(X) }\OtherTok{\textless{}{-}}\NormalTok{ uniq\_var}
  
  \ControlFlowTok{if}\NormalTok{(is1}\SpecialCharTok{==}\ConstantTok{FALSE}\NormalTok{)\{}
\NormalTok{    X }\OtherTok{=}\NormalTok{ X[,}\SpecialCharTok{{-}}\DecValTok{1}\NormalTok{]}
\NormalTok{  \}}
  \FunctionTok{return}\NormalTok{(}\FunctionTok{myOLS}\NormalTok{(}\AttributeTok{X=}\NormalTok{X, }\AttributeTok{Y=}\NormalTok{Y, }\AttributeTok{is1=}\NormalTok{is1))}
\NormalTok{\}}
\end{Highlighting}
\end{Shaded}

\begin{Shaded}
\begin{Highlighting}[]
\NormalTok{n }\OtherTok{=} \DecValTok{30}
\FunctionTok{set.seed}\NormalTok{(}\DecValTok{0}\NormalTok{)}
\NormalTok{XF }\OtherTok{=} \FunctionTok{rep}\NormalTok{(}\FunctionTok{c}\NormalTok{(}\StringTok{"A"}\NormalTok{,}\StringTok{"B"}\NormalTok{,}\StringTok{"C"}\NormalTok{),}\AttributeTok{each=}\DecValTok{10}\NormalTok{)}
\NormalTok{Y }\OtherTok{=} \FunctionTok{rnorm}\NormalTok{(n) }\SpecialCharTok{+} \FunctionTok{rep}\NormalTok{(}\FunctionTok{c}\NormalTok{(}\DecValTok{1}\NormalTok{,}\DecValTok{2}\NormalTok{,}\DecValTok{3}\NormalTok{),}\AttributeTok{each=}\DecValTok{10}\NormalTok{)}
\NormalTok{fit1 }\OtherTok{=} \FunctionTok{lm}\NormalTok{(Y }\SpecialCharTok{\textasciitilde{}}\NormalTok{ XF)}
\FunctionTok{summary}\NormalTok{(fit1) }\CommentTok{\# with an intercept}
\end{Highlighting}
\end{Shaded}

\begin{verbatim}
## 
## Call:
## lm(formula = Y ~ XF)
## 
## Residuals:
##      Min       1Q   Median       3Q      Max 
## -1.89887 -0.62259  0.01652  0.70476  2.04573 
## 
## Coefficients:
##             Estimate Std. Error t value Pr(>|t|)    
## (Intercept)   1.3589     0.2828   4.805 5.15e-05 ***
## XFB           0.2786     0.4000   0.697 0.492057    
## XFC           1.7105     0.4000   4.276 0.000212 ***
## ---
## Signif. codes:  0 '***' 0.001 '**' 0.01 '*' 0.05 '.' 0.1 ' ' 1
## 
## Residual standard error: 0.8944 on 27 degrees of freedom
## Multiple R-squared:  0.4382, Adjusted R-squared:  0.3966 
## F-statistic: 10.53 on 2 and 27 DF,  p-value: 0.0004163
\end{verbatim}

\begin{Shaded}
\begin{Highlighting}[]
\FunctionTok{myAnova1}\NormalTok{(Y, XF, }\AttributeTok{is1=}\ConstantTok{FALSE}\NormalTok{)}
\end{Highlighting}
\end{Shaded}

\begin{verbatim}
## [[1]]
##         [,1]
## X0 1.3589240
## B  0.2785947
## C  1.7104858
## 
## [[2]]
##        X0         B         C 
## 0.2777328 0.3927735 0.3927735
\end{verbatim}

\begin{Shaded}
\begin{Highlighting}[]
\NormalTok{fit0 }\OtherTok{=} \FunctionTok{lm}\NormalTok{(Y }\SpecialCharTok{\textasciitilde{}} \SpecialCharTok{{-}}\DecValTok{1} \SpecialCharTok{+}\NormalTok{ XF)}
\FunctionTok{summary}\NormalTok{(fit0) }\CommentTok{\# without an intercept}
\end{Highlighting}
\end{Shaded}

\begin{verbatim}
## 
## Call:
## lm(formula = Y ~ -1 + XF)
## 
## Residuals:
##      Min       1Q   Median       3Q      Max 
## -1.89887 -0.62259  0.01652  0.70476  2.04573 
## 
## Coefficients:
##     Estimate Std. Error t value Pr(>|t|)    
## XFA   1.3589     0.2828   4.805 5.15e-05 ***
## XFB   1.6375     0.2828   5.790 3.69e-06 ***
## XFC   3.0694     0.2828  10.853 2.39e-11 ***
## ---
## Signif. codes:  0 '***' 0.001 '**' 0.01 '*' 0.05 '.' 0.1 ' ' 1
## 
## Residual standard error: 0.8944 on 27 degrees of freedom
## Multiple R-squared:  0.8659, Adjusted R-squared:  0.851 
## F-statistic: 58.13 on 3 and 27 DF,  p-value: 6.599e-12
\end{verbatim}

\begin{Shaded}
\begin{Highlighting}[]
\FunctionTok{myAnova1}\NormalTok{(Y, XF, }\AttributeTok{is1=}\ConstantTok{TRUE}\NormalTok{)}
\end{Highlighting}
\end{Shaded}

\begin{verbatim}
## [[1]]
##       [,1]
## A 1.358924
## B 1.637519
## C 3.069410
## 
## [[2]]
##         A         B         C 
## 0.2828292 0.2828292 0.2828292
\end{verbatim}

\hypertarget{problem-4}{%
\section{Problem 4}\label{problem-4}}

\begin{Shaded}
\begin{Highlighting}[]
\NormalTok{n }\OtherTok{=} \DecValTok{30} 
\FunctionTok{set.seed}\NormalTok{(}\DecValTok{0}\NormalTok{)}
\NormalTok{X }\OtherTok{=} \FunctionTok{runif}\NormalTok{(n)}\SpecialCharTok{*}\DecValTok{4{-}2}
\NormalTok{Y }\OtherTok{=} \DecValTok{1} \SpecialCharTok{+} \DecValTok{3}\SpecialCharTok{*}\NormalTok{X }\SpecialCharTok{+} \FunctionTok{rnorm}\NormalTok{(n)}
\NormalTok{fit1 }\OtherTok{=} \FunctionTok{lm}\NormalTok{(Y }\SpecialCharTok{\textasciitilde{}}\NormalTok{ X) }
\FunctionTok{summary}\NormalTok{(fit1) }\CommentTok{\# beta0\_hat = 1.0161; beta1\_hat = 2.9304}
\end{Highlighting}
\end{Shaded}

\begin{verbatim}
## 
## Call:
## lm(formula = Y ~ X)
## 
## Residuals:
##     Min      1Q  Median      3Q     Max 
## -1.2486 -0.5518 -0.1069  0.5028  1.6770 
## 
## Coefficients:
##             Estimate Std. Error t value Pr(>|t|)    
## (Intercept)   1.0161     0.1480   6.867 1.84e-07 ***
## X             2.9304     0.1241  23.605  < 2e-16 ***
## ---
## Signif. codes:  0 '***' 0.001 '**' 0.01 '*' 0.05 '.' 0.1 ' ' 1
## 
## Residual standard error: 0.8068 on 28 degrees of freedom
## Multiple R-squared:  0.9522, Adjusted R-squared:  0.9504 
## F-statistic: 557.2 on 1 and 28 DF,  p-value: < 2.2e-16
\end{verbatim}

\hypertarget{section-3}{%
\subsection{4.1}\label{section-3}}

\begin{Shaded}
\begin{Highlighting}[]
\NormalTok{myFullObj }\OtherTok{\textless{}{-}} \ControlFlowTok{function}\NormalTok{(par) \{}
\CommentTok{\# Inputs:}
\CommentTok{\# * b is the vector of regression coefficients, b=[b0,b1];}
\CommentTok{\# * sig is the standard dev of errors; sig \textgreater{} 0}
\CommentTok{\# Output: the negative log{-}likelihood of the observed data (Y given X) evaluated at b and sig}
\NormalTok{  b }\OtherTok{=}\NormalTok{ par[}\DecValTok{1}\SpecialCharTok{:}\DecValTok{2}\NormalTok{]}
\NormalTok{  sig }\OtherTok{=}\NormalTok{ par[}\DecValTok{3}\NormalTok{]}
\NormalTok{  miu }\OtherTok{\textless{}{-}}\NormalTok{ b[}\DecValTok{1}\NormalTok{]}\SpecialCharTok{+}\FunctionTok{mean}\NormalTok{(X)}\SpecialCharTok{*}\NormalTok{b[}\DecValTok{2}\NormalTok{]}
  \FunctionTok{return}\NormalTok{((}\SpecialCharTok{{-}}\DecValTok{1}\NormalTok{)}\SpecialCharTok{*}\FunctionTok{sum}\NormalTok{(}\FunctionTok{dnorm}\NormalTok{(Y, }\AttributeTok{mean=}\NormalTok{miu, }\AttributeTok{sd =}\NormalTok{ sig, }\AttributeTok{log=}\ConstantTok{TRUE}\NormalTok{)))}
\NormalTok{\}}
\end{Highlighting}
\end{Shaded}

\hypertarget{section-4}{%
\subsection{4.2}\label{section-4}}

\begin{Shaded}
\begin{Highlighting}[]
\ControlFlowTok{for}\NormalTok{(i }\ControlFlowTok{in} \FunctionTok{c}\NormalTok{(}\DecValTok{2}\NormalTok{,}\FloatTok{0.1}\NormalTok{,}\DecValTok{1}\NormalTok{,}\DecValTok{10}\NormalTok{,}\DecValTok{100}\NormalTok{))\{}
\NormalTok{  sigKnown }\OtherTok{=}\NormalTok{ i}
\NormalTok{  myObj1 }\OtherTok{\textless{}{-}} \ControlFlowTok{function}\NormalTok{(b)\{}
    \FunctionTok{myFullObj}\NormalTok{(}\FunctionTok{c}\NormalTok{(b,sigKnown))}
\NormalTok{  \}}
  \FunctionTok{print}\NormalTok{(}\FunctionTok{c}\NormalTok{(}\StringTok{"Sigknown = "}\NormalTok{, i))}
  \FunctionTok{print}\NormalTok{(}\FunctionTok{optim}\NormalTok{(}\AttributeTok{par=}\FunctionTok{c}\NormalTok{(}\DecValTok{1}\NormalTok{,}\DecValTok{3}\NormalTok{),myObj1,}\AttributeTok{method=}\StringTok{"BFGS"}\NormalTok{))}
  \FunctionTok{print}\NormalTok{(}\StringTok{"{-}{-}{-}{-}{-}{-}{-}{-}{-}{-}{-}{-}{-}{-}{-}{-}"}\NormalTok{)}
\NormalTok{\}}
\end{Highlighting}
\end{Shaded}

\begin{verbatim}
## [1] "Sigknown = " "2"          
## $par
## [1] 1.008136 3.000914
## 
## $value
## [1] 95.97173
## 
## $counts
## function gradient 
##        4        3 
## 
## $convergence
## [1] 0
## 
## $message
## NULL
## 
## [1] "----------------"
## [1] "Sigknown = " "0.1"        
## $par
## [1] 1.008136 3.000914
## 
## $value
## [1] 19002.15
## 
## $counts
## function gradient 
##       20        2 
## 
## $convergence
## [1] 0
## 
## $message
## NULL
## 
## [1] "----------------"
## [1] "Sigknown = " "1"          
## $par
## [1] 1.008136 3.000914
## 
## $value
## [1] 218.0048
## 
## $counts
## function gradient 
##        5        3 
## 
## $convergence
## [1] 0
## 
## $message
## NULL
## 
## [1] "----------------"
## [1] "Sigknown = " "10"         
## $par
## [1] 1.008136 3.000914
## 
## $value
## [1] 98.55008
## 
## $counts
## function gradient 
##        3        3 
## 
## $convergence
## [1] 0
## 
## $message
## NULL
## 
## [1] "----------------"
## [1] "Sigknown = " "100"        
## $par
## [1] 1.000025 3.000003
## 
## $value
## [1] 165.7423
## 
## $counts
## function gradient 
##        2        1 
## 
## $convergence
## [1] 0
## 
## $message
## NULL
## 
## [1] "----------------"
\end{verbatim}

For all the tested values of sigma (0.1, 1, 10, 100) it produces very
similar results with beta0\_hat always close to 1 and beta1\_hat always
very close to 3. The negative log-likelihood for each of these differ
though. This means that no matter the standard deviation it is always
able to minimize the function to the same beta parameters.

These results are very similar to the ones produced by the lm()
function. This means that both methods converge at a similar answer for
a linear regression model that describes the data. This is likely
because the estimated line is the best fit to the data.

\hypertarget{section-5}{%
\subsection{4.3}\label{section-5}}

\begin{Shaded}
\begin{Highlighting}[]
\FunctionTok{optim}\NormalTok{(}\AttributeTok{par =} \FunctionTok{c}\NormalTok{(}\DecValTok{1}\NormalTok{,}\DecValTok{3}\NormalTok{,}\DecValTok{0}\NormalTok{), myFullObj,}\AttributeTok{method=}\StringTok{"L{-}BFGS{-}B"}\NormalTok{,}\AttributeTok{lower=}\FunctionTok{c}\NormalTok{(}\SpecialCharTok{{-}}\ConstantTok{Inf}\NormalTok{, }\SpecialCharTok{{-}}\ConstantTok{Inf}\NormalTok{, }\DecValTok{10}\SpecialCharTok{\^{}}\NormalTok{(}\SpecialCharTok{{-}}\DecValTok{5}\NormalTok{)))}
\end{Highlighting}
\end{Shaded}

\begin{verbatim}
## $par
## [1] 1.008170 3.000571 3.563113
## 
## $value
## [1] 80.6872
## 
## $counts
## function gradient 
##       32       32 
## 
## $convergence
## [1] 0
## 
## $message
## [1] "CONVERGENCE: REL_REDUCTION_OF_F <= FACTR*EPSMCH"
\end{verbatim}

For different values of beta0\_hat and beta1\_hat the sigma value stays
constant at approximately 3.56. When optim() is initialized with
different values of beta0\_hat and beta1\_hat it converges at different
estimates of beta but the sigma estimate stays the same. This means that
there are multiple values of beta0\_hat and beta1\_hat that are optimal
but only one optimum for sigma.

\end{document}
