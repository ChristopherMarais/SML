% Options for packages loaded elsewhere
\PassOptionsToPackage{unicode}{hyperref}
\PassOptionsToPackage{hyphens}{url}
%
\documentclass[
  11pt,
]{article}
\usepackage{amsmath,amssymb}
\usepackage{lmodern}
\usepackage{iftex}
\ifPDFTeX
  \usepackage[T1]{fontenc}
  \usepackage[utf8]{inputenc}
  \usepackage{textcomp} % provide euro and other symbols
\else % if luatex or xetex
  \usepackage{unicode-math}
  \defaultfontfeatures{Scale=MatchLowercase}
  \defaultfontfeatures[\rmfamily]{Ligatures=TeX,Scale=1}
\fi
% Use upquote if available, for straight quotes in verbatim environments
\IfFileExists{upquote.sty}{\usepackage{upquote}}{}
\IfFileExists{microtype.sty}{% use microtype if available
  \usepackage[]{microtype}
  \UseMicrotypeSet[protrusion]{basicmath} % disable protrusion for tt fonts
}{}
\makeatletter
\@ifundefined{KOMAClassName}{% if non-KOMA class
  \IfFileExists{parskip.sty}{%
    \usepackage{parskip}
  }{% else
    \setlength{\parindent}{0pt}
    \setlength{\parskip}{6pt plus 2pt minus 1pt}}
}{% if KOMA class
  \KOMAoptions{parskip=half}}
\makeatother
\usepackage{xcolor}
\usepackage[margin=2cm]{geometry}
\usepackage{graphicx}
\makeatletter
\def\maxwidth{\ifdim\Gin@nat@width>\linewidth\linewidth\else\Gin@nat@width\fi}
\def\maxheight{\ifdim\Gin@nat@height>\textheight\textheight\else\Gin@nat@height\fi}
\makeatother
% Scale images if necessary, so that they will not overflow the page
% margins by default, and it is still possible to overwrite the defaults
% using explicit options in \includegraphics[width, height, ...]{}
\setkeys{Gin}{width=\maxwidth,height=\maxheight,keepaspectratio}
% Set default figure placement to htbp
\makeatletter
\def\fps@figure{htbp}
\makeatother
\setlength{\emergencystretch}{3em} % prevent overfull lines
\providecommand{\tightlist}{%
  \setlength{\itemsep}{0pt}\setlength{\parskip}{0pt}}
\setcounter{secnumdepth}{-\maxdimen} % remove section numbering
\ifLuaTeX
  \usepackage{selnolig}  % disable illegal ligatures
\fi
\IfFileExists{bookmark.sty}{\usepackage{bookmark}}{\usepackage{hyperref}}
\IfFileExists{xurl.sty}{\usepackage{xurl}}{} % add URL line breaks if available
\urlstyle{same} % disable monospaced font for URLs
\hypersetup{
  pdftitle={SML HW2},
  pdfauthor={Christopher Marais},
  hidelinks,
  pdfcreator={LaTeX via pandoc}}

\title{SML HW2}
\author{Christopher Marais}
\date{}

\begin{document}
\maketitle

\hypertarget{chapter-3}{%
\subsection{Chapter 3:}\label{chapter-3}}

\hypertarget{question-4}{%
\subsubsection{Question 4}\label{question-4}}

4.1

\hypertarget{required-typed-problems}{%
\section{Required Typed Problems}\label{required-typed-problems}}

\textbf{Typed Problem 1.}

Let \(Y_1,...,Y_n\) be iid rvs with \(E(Y_i)=a\) and \(E(Y_i^2)=b\), so
that \(Var(Y_i)=b-a^2\).

Define \(T = \sum_{i=1}^n (Y_i-\bar{Y})^2\), where
\(\bar{Y} = n^{-1}\sum_{i=1}^n Y_i\) is the sample mean.

1.1. (Optional) Use the properties/calculus of expectations to find
\(E(T)\). If you are not able to find \(E(T)\), you can use use
\(E(T) = (n-1)Var(Y_i)\) in subsequent subproblems.

1.2. Suppose we estimate the population variance \(Var(Y_i)\) by \(cT\)
for some constant \(c>0\). What value of \(c\) results in an unbiased
estimator of the population variance? (The answer you should get is
\(c=1/(n-1)\).) Let \(T_1 = cT\) be this unbiased estimator.

1.3. Let \(Y_1,...,Y_n\) be iid \(Normal(\mu,\sigma^2)\), where \(\mu\)
and \(\sigma^2\) are the population mean and variance, respectively. One
can show that \(T_2 = T/n\) is the MLE for \(\sigma^2\); you can take
this fact for granted.

Use R to examine the small-sample properties of \(T_1\) and \(T_2\) as
follows:

\begin{enumerate}
\def\labelenumi{(\alph{enumi})}
\item
  Generate the data as follows:

  \begin{verbatim}
  m=1000; n=4; # n is the sample size; m is the # of replications
  set.seed(0); 
  M = matrix(rnorm(m*n),nrow=m); # default parameters in rnorm are mean=0, sd=1;
  # M is an m-by-n matrix with replications of the experiment stored in rows
  \end{verbatim}
\item
  For each row of M, evaluate and store values of \(T_1\) and \(T_2\),
  in separate vectors. (Optional): you can do this without loops using
  apply() function
\item
  Plot histograms of \(T_1\) and \(T_2\).
\item
  ``Monte Carlo integration'' is estimation of population moments of a
  rv \(X\) by the corresponding sample moments whenever one can simulate
  iid variates \(X_1,X_2,\ldots\) from the sampling distribution of
  \(X\). I.e., using the law of large numbers (and another result known
  as the continuous mapping theorem) \(\bar{X}_n \rightarrow E(X)\) and
  \(S_n^2 \rightarrow Var(X)\) as \(n\rightarrow \infty\), where
  \(\bar{X}_n\) and \(S_n^2\) are the sample mean and the sample
  variance, respectively. Use ``Monte Carlo integration'' to estimate
  bias, variance and MSE of the two estimators. Specifically, you can
  estimate \(E(T_1)\) and \(E(T_2)\) using the respective sample means,
  and (population) variances of \(T_1\) and \(T_2\) using the sample
  variances of \(T_1\) and \(T_2\).
\end{enumerate}

Briefly discuss your findings in (c) and (d).

1.4. Suppose we are now interested in the population standard deviation,
i.e., \(\sigma=\sqrt{\sigma^2}\). Explain/argue whether \(\sqrt{T_1}\)
is unbiased for estimation of \(\sigma\), and why. Feel free to extend
the simulation study in 1.3 to reinforce your answer.

\end{document}
